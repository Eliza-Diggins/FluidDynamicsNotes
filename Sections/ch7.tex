\section{Intuition}

Supersonic flows occur when a \textbf{disturbance propagates faster than the local speed of sound}. In this regime, information cannot propagate upstream to ``warn'' the undisturbed medium, and discontinuities in the flow (shock waves) form.
\medskip

\noindent
Consider the following simple scenario: you (the observer) sit at the origin $(0,0)$, while a uniform flow with velocity $v$ and sound speed $c_s$ moves past you in the $+x$ direction. At some time $t_0 = 0$, you insert a tuning fork into the flow, generating periodic disturbances with period $\tau$.
\begin{remark}
    Intuitively, if the disturbance speed is much smaller than the sound speed ($v \ll c_s$), then the tuning fork generates small-amplitude density and pressure waves that propagate outward through the fluid at speed $c_s$ in all directions (in the fluid frame). Because these waves travel much faster than the source itself, they can reach the upstream fluid before the source arrives, giving the medium time to adjust smoothly. As a result, successive wavefronts remain separated, and the flow changes continuously without sharp discontinuities.

    In contrast, if the disturbance speed is much greater than the sound speed ($v \gg c_s$), the fluid upstream cannot receive any ``advance notice'' of the approaching source. The chain of particle collisions that transmits pressure changes cannot outpace the source, so the medium only reacts when the source is already upon it. This leads to a pile-up of wavefronts into a narrow region where the density, pressure, and velocity change abruptly --- a shock front. In the supersonic regime, the envelope of these piled-up wavefronts forms the Mach cone downstream of the disturbance.
\end{remark}

In the frame of the fluid, each disturbance propagates spherically with speed $c_s$. The center of the $k$-th disturbance in the lab frame is located at:
\[
x_\mathrm{center} = k\,\tau\, v .
\]
Thus, the equation for the $k$-th wavefront is
\begin{equation}
    (x - k\tau v)^2 + y^2 = c_s^2 \, (t - k\tau)^2 .
\end{equation}
At the downstream edge of the $k$-th wavefront, we have:
\begin{align}
    x - k\tau v &= c_s \, (t - k\tau) , \\
    \implies x_k &= c_s t + k\tau (v - c_s) .
\end{align}
The longitudinal spacing between successive wavefronts is therefore:
\begin{equation}
    \Delta x = x_{k+1} - x_k = \tau (v - c_s) .
\end{equation}

We can now classify the flow regimes:
\begin{enumerate}
    \item \textbf{Subsonic} ($v < c_s$): Wavefronts remain in order; disturbances can propagate upstream.
    \item \textbf{Sonic} ($v = c_s$): Wavefronts ``pile up'' downstream, producing a stationary compression region.
    \item \textbf{Supersonic} ($v > c_s$): Successive wavefronts overtake one another downstream, forming a shock front.
\end{enumerate}


\subsection{Mach Cone Formation}

In the supersonic case ($v > c_s$), the envelope of all wavefronts forms a conical shock surface,the \emph{Mach cone} --- that trails the disturbance in the downstream direction.

At time $t$, the furthest extent of any disturbance in the $y$-direction is
\[
y_{\max} = c_s t ,
\]
while the furthest downstream extent in $x$ is
\[
x_{\max} = v t .
\]
The half-opening angle $\theta$ of the Mach cone is therefore:
\begin{equation}
    \sin\theta = \frac{c_s}{v} .
\end{equation}

\begin{definition}[Mach Number]
The \emph{Mach number} $M$ is the ratio of the object's speed to the local speed of sound:
\[
M \equiv \frac{v}{c_s} .
\]
In terms of $M$, the Mach angle is:
\[
\theta = \sin^{-1} \left( \frac{1}{M} \right) .
\]
\end{definition}

A larger Mach number corresponds to a narrower cone, while $M \to 1^+$ corresponds to a very wide, weak cone.

\section{The Rankine-Hugoniot Conditions}

Consider a \textbf{shock front} dividing two regions of fluid. We refer to each side as the \textbf{upstream} and \textbf{downstream} side of the shock front. For the sake of simplicity, we assume the front occurs at $x = 0$ in some reference frame and that, on either side of the front, we have some $\rho_{1,2},p_{1,2}, \;\text{and}\; u_{1,2}$. Now, certain conservation laws must still be true, namely those which provide us with the \textbf{Euler Equations}. As such, we can define some elements of the behavior across the shock front in terms of these conservation rules.
\begin{remark}
    An \textbf{critical} realization is that the Rankine-Hugoniot relations which we are soon to derive are  valid \textbf{only in the rest frame of the shock}. As such, we are always implicitly performing a Galilean transformation into that frame when we use them. 
    \par
    In many cases, this makes the intuition for which side of the shock is which tricky: if a bow shock is driven into the ICM of a galaxy cluster, the gas in the galaxy cluster is the \textbf{upstream side} since it moves towards the shock in the shock's reference frame. 
\end{remark}

\subsection*{The Continuity Condition}
In Eulerian form the continuity equation is,
\[
\frac{\partial \rho}{\partial t} + \nabla \cdot (\rho u) = 0.
\]
if we integrate across some infinitesmal width $\delta x$ on either side of the shock front,
\[
\frac{\partial}{\partial t} \int_{-\delta x}^{\delta x} \rho \;dx + (\rho u)_{\delta x} - (\rho u)_{-\delta x} =0.
\]
Now, as $\delta x \to 0$, we clearly have that the integral term vanishes and
\[
\boxed{\rho_1u_1 = \rho_2u_2}.
\]

\subsection*{The Momentum Condition}
In one-dimensional inviscid flow with an external body force ${\bf f}_{\rm ext}$, the momentum equation in \emph{conservative form} is
\begin{equation}
\frac{\partial (\rho u)}{\partial t} 
+ \frac{\partial}{\partial x} \left( \rho u^2 + p \right)
= \rho {\bf f}_{\rm ext}.
\end{equation}
Here $\rho u$ is the momentum density, and $\rho u^2 + p$ is the momentum flux (mass flux of momentum plus the pressure force).

\medskip

We now integrate this equation across a thin control volume enclosing a discontinuity at $x = 0$, extending from $x = -\delta x$ to $x = +\delta x$. In the shock rest frame (steady state), the time derivative vanishes upon integration:
\begin{equation}
\int_{-\delta x}^{+\delta x} 
\frac{\partial}{\partial x} \left( \rho u^2 + p \right) dx
= \int_{-\delta x}^{+\delta x} \rho {\bf f}_{\rm ext} \, dx.
\end{equation}

If ${\bf f}_{\rm ext}$ is bounded, its contribution is $O(\delta x)$ and vanishes as $\delta x \to 0$. Therefore, in the limit we obtain
\begin{equation}
\left[ \rho u^2 + p \right]_{1}^{2} = 0,
\end{equation}
where $[A]_1^2 \equiv A_2 - A_1$ denotes the jump across the discontinuity. This is the \textbf{momentum Rankine–Hugoniot condition}:
\begin{equation}
\boxed{
\rho_1 u_1^2 + p_1 \;=\; \rho_2 u_2^2 + p_2.}
\end{equation}
An equivalent statement may be created where the force is not bounded. See, for example, Thorne+Blandford.

\subsection*{The Energy Condition}

To derive the relevant condition for energy conservation across a shock, we adopt two simplifying assumptions:

\begin{enumerate}
    \item \textbf{Adiabatic flow}: there is no external heating or cooling, so we may ignore source terms in the energy equation.
    \item \textbf{Inviscid flow}: viscous dissipation is neglected.
\end{enumerate}

\noindent
Under these assumptions, the energy equation takes the form
\[
\frac{\partial E}{\partial t} + \nabla \cdot \left[(E+p)\,\mathbf{u}\right] = 0,
\]
where $E = \tfrac{1}{2}\rho u^2 + \rho \epsilon$ is the total energy density, with $\epsilon$ the specific internal energy.  

\medskip
This equation can also be expressed in terms of enthalpy, $h \equiv \epsilon + p/\rho$, as
\[
\frac{\partial E}{\partial t} + \nabla \cdot \left[\left(\tfrac{1}{2}\rho u^2 + \rho h\right)\mathbf{u}\right] = 0.
\]
\medskip

Assuming steady state and integrating across a vanishingly thin control volume that encloses the shock, we obtain equivalent jump conditions:
\begin{equation}
    \begin{aligned}
        \big[u(E+p)\big]_1^2 &= 0 
        &&\;\;\; \text{(flux of total energy + pressure)} \\[6pt]
        \big[u(\tfrac{1}{2}\rho u^2 + \rho h)\big]_1^2 &= 0
        &&\;\;\; \text{(flux of kinetic + enthalpy energy)} .
    \end{aligned}
\end{equation}

\noindent
Finally, using the mass conservation condition $\rho u =$ const across the shock, we can eliminate $u$ and write the energy condition in more compact forms:
\begin{equation}
    \boxed{ \;\;\left[ \mathcal{E} + \tfrac{p}{\rho} \right]_1^2 = 0 \;\;} ,
    \qquad\text{or equivalently}\qquad
    \boxed{ \;\;\left[ \tfrac{1}{2}u^2 + h \right]_1^2 = 0 \;\;} .
\end{equation}
Here $\mathcal{E} = \epsilon + \tfrac{1}{2}u^2$ is the specific total energy.

\subsection{The Rankine--Hugoniot Conditions}

\begin{definition}[Rankine--Hugoniot Conditions]
The \emph{Rankine--Hugoniot (RH) conditions} express the conservation of mass, momentum, and energy across a steady, planar shock front.  
They are valid only in the \textbf{rest frame of the shock}, where the discontinuity is stationary and the upstream fluid flows into the front.  
Together, they constrain the allowed discontinuities in density, velocity, pressure, and temperature.

\medskip
For upstream quantities $(\rho_1, u_1, p_1)$ and downstream quantities $(\rho_2, u_2, p_2)$, the RH conditions are:

\begin{enumerate}
    \item \textbf{Mass conservation (continuity):}
\begin{equation}
    \label{eq:rh_continuity}
    \rho_1u_1 = \rho_2 u_2
\end{equation}
    The mass flux through the shock is the same on both sides.

    \item \textbf{Momentum conservation:}
    \begin{equation}
   \label{eq:rh_momentum}
                \rho_1 u_1^2 + p_1 \;=\; \rho_2 u_2^2 + p_2 .
    \end{equation}
    The sum of momentum flux and pressure force is conserved.

    \item \textbf{Energy conservation:}
    \begin{equation}
        \label{eq:rh_energy}
        \frac{1}{2}u_1^2 + h_1 \;=\; \frac{1}{2}u_2^2 + h_2 ,
    \end{equation}
    where $h = \epsilon + p/\rho$ is the specific enthalpy.  
    This states that the specific total energy (kinetic plus thermal) is continuous across the shock.
\end{enumerate}

\noindent
Together, these three relations define the \emph{Rankine--Hugoniot conditions}.  
They show that shocks are not arbitrary discontinuities: only those jumps that satisfy mass, momentum, and energy conservation are physically possible.
\end{definition}


\subsection{Additional Forms of the RH Conditions}

Now, in their current form, equations~\ref{eq:rh_energy}, \eqref{eq:rh_momentum}, and \eqref{eq:rh_energy} are dependent on the flow velocity, the internal energy, and the density / pressure. In many scenarios,\textbf{ these are not all measurable properties of the flow} and instead we seek to find a simpler / more useful way to cast these relationships. The first step in doing so is to better understand the internal energy $E$ which appears in the above equations. Let us formally \textbf{assume} a polytropic equation of state of the form,
\[
p = \rho^\Gamma,
\]
The convenience of this assumption is that the enthalpy is 
\[
h = \int^P \frac{dP}{\rho} = \int^\rho \Gamma \rho^{\Gamma -2} \;d \rho = \frac{\Gamma}{\Gamma -1} \rho^{\Gamma -1}.
\]
Noting that
\[
c_s^2 = \frac{\partial P}{\partial \rho} = \Gamma \rho^{\Gamma -1} \implies \boxed{h = \frac{c_s^2}{\Gamma -1}.}
\]
We can therefore get the very convenient form of the energy condition \eqref{eq:rh_energy}:
\begin{equation}
    \label{eq:rh_energy_enth}
    \boxed{
    \frac{1}{2}u_1^2 + \frac{c_1^2}{\Gamma -1} = \frac{1}{2}u_2^2 +\frac{c_2^2}{\Gamma -1}
    }
\end{equation}
This form of the \textbf{Rankine-Huginiot condition} is already quite nice, but there are many manipulations to be made to these expressions in order to get various forms worth exploration. At this stage, it is worth developing something of a heuristic picture of how these can be used.
\par
We have, in this form of the RH conditions, 4 sets of variables: $\rho_{[1,2]}$, $u_{[1,2]}$, $c_{[1,2]}^2$, and $P_{[1,2]}$. Now, the RH conditions (and the EOS) provide the following relationships:
\vspace{0.5cm}
\begin{enumerate}
    \item \textbf{Continuity}: Relates $\rho$ and $u$.
    \item \textbf{Momentum}: Relates $P$ and $u$.
    \item \textbf{Energy}: Relates $u$ and $c_s^2$.
    \item \textbf{EOS}: Relates $P$ and $\rho$.
\end{enumerate}
\vspace{0.5cm}
\subsubsection*{Mach Number Form of the RH Conditions}

It is often useful to re–express the Rankine--Hugoniot conditions in terms of the
\textbf{Mach number},
\[
M \equiv \frac{u}{c_s},
\]
which combines the flow speed $u$ and the sound speed $c_s$ into a single dimensionless variable.  
This is particularly helpful in astrophysical contexts, where shocks are commonly characterized by their upstream Mach number $M_1$.

\medskip

Starting from the energy condition across the shock, we can write
\[
\frac{1}{2} + \frac{M_1^{-2}}{\Gamma -1}
= \frac{1}{2}\frac{u_2^2}{u_1^2} + \frac{(c_2^2/u_1^2)}{\Gamma -1},
\]
where $\Gamma$ is the adiabatic index.  

\medskip

From the continuity condition (\eqref{eq:rh_continuity}) we know that
\[
\frac{\rho_1}{\rho_2} = \frac{u_2}{u_1} \equiv x ,
\]
so that
\[
\frac{1}{2} + \frac{M_1^{-2}}{\Gamma -1}
= \frac{1}{2}x^2 + \frac{c_2^2}{c_1^2}\frac{M_1^{-2}}{\Gamma -1}.
\]
Rearranging gives
\[
\frac{1}{2}(1-x^2) = \frac{M_1^{-2}}{\Gamma -1}\left(\frac{c_2^2}{c_1^2} - 1\right).
\]

\medskip

The ratio of sound speeds follows from the equation of state:
\[
\frac{c_2^2}{c_1^2} = \frac{P_2}{P_1}\frac{\rho_1}{\rho_2} = \frac{P_2}{P_1}x.
\]

To eliminate $P_2/P_1$, we use the momentum Rankine--Hugoniot condition:
\[
\rho_1 u_1^2 + P_1 = \rho_2 u_2^2 + P_2.
\]
Dividing through by $P_1$ and using $x \equiv u_2/u_1 = \rho_1/\rho_2$, we can rewrite the right-hand side:
\[
\frac{\rho_2 u_2^2}{P_1} = \frac{\rho_2 (x u_1)^2}{P_1}
= \frac{\rho_1 u_1^2}{P_1}\,x.
\]
\noindent
Therefore the momentum condition becomes
\[
\frac{\rho_1 u_1^2}{P_1} + 1 = \frac{\rho_1 u_1^2}{P_1}\,x + \frac{P_2}{P_1}.
\]

\noindent
Rearranging gives
\[
1 - \frac{P_2}{P_1} = \frac{\rho_1 u_1^2}{P_1}(x-1).
\]
Next, we express the prefactor in terms of the Mach number. Since
\[
M_1^2 = \frac{u_1^2}{c_1^2}, \qquad c_1^2 = \frac{\Gamma P_1}{\rho_1},
\]
we have
\[
\frac{\rho_1 u_1^2}{P_1} = \frac{\rho_1}{P_1} M_1^2 c_1^2
= \frac{\rho_1}{P_1} M_1^2 \frac{\Gamma P_1}{\rho_1}
= \Gamma M_1^2.
\]

\noindent
Thus the pressure jump condition is
\[
\frac{P_2}{P_1} = 1 + \Gamma M_1^2(1-x).
\]
Finally, recalling that $c^2 = \Gamma P/\rho$, the sound speed ratio may be expressed as
\[
\frac{c_2^2}{c_1^2} = \frac{P_2}{P_1}\frac{\rho_1}{\rho_2}
= \frac{P_2}{P_1}\,x
= x\left[1 + \Gamma M_1^2(1-x)\right].
\]
Substituting this into the modified energy equation yields
\[
\frac{M_1^2}{2}(x^2-1) = \frac{1}{\Gamma -1}\left[\Gamma x(1-x)M_1^{2} + x - 1\right].
\]

Factoring out $(x-1)$ from the modified energy equation and simplifying gives
\[
\frac{1}{2}(\Gamma -1) M_1^2 (x+1) + \Gamma x - M_1^2 = 0.
\]
Solving this quadratic relation for $x = \rho_1/\rho_2$ and inverting, we obtain the 
classic \textbf{compression ratio} across a shock:
\begin{equation}
\label{eq:rh_density_ratio_from_mach}
\boxed{\;\;\frac{\rho_2}{\rho_1} \;=\; \frac{(\Gamma+1)M_1^2}{(\Gamma-1)M_1^2 + 2}\;\;}
\end{equation}
\noindent
This compact expression shows that the density jump depends only on the upstream Mach number.  
In the strong shock limit ($M_1 \to \infty$), the ratio saturates at
$\rho_2/\rho_1 = (\Gamma+1)/(\Gamma-1)$, which equals \framebox{$4$ for a monatomic ideal gas ($\Gamma = 5/3$)}. As the shock weakens, the density ratio becomes $1$.
\par
From the derivation above, we can also write down the \textbf{pressure ratio} in the form
\begin{equation}
\boxed{\;\;\frac{P_2}{P_1} = 1 + \frac{2\Gamma}{\Gamma+1}\,(M_1^2 - 1)\;\;},
\end{equation}
where we have used the fact that $x = u_2/u_1 = \rho_1/\rho_2$ and the expression we derived from $P_2/P_1$ in terms of $x$ above.
\par

\subsubsection*{The Density-Pressure Form}
It is sometimes convenient to eliminate the Mach number entirely and express
the compression ratio $\rho_2/\rho_1$ directly in terms of the pressure ratio
$P_2/P_1$. Starting from the momentum condition in dimensionless form,
\[
\frac{P_2}{P_1} = 1 + \Gamma M_1^2 (1 - x),
\]
with $x \equiv \rho_1/\rho_2$, we can rearrange this relation to isolate $M_1^2$:
\[
M_1^2 = \frac{ \tfrac{P_2}{P_1} - 1 }{ \Gamma(1 - x) }.
\]
On the other hand, from the density ratio expressed in terms of Mach number
(eq.~\ref{eq:rh_density_ratio_from_mach}),
\[
\frac{\rho_2}{\rho_1} = \frac{(\Gamma+1) M_1^2}{(\Gamma-1)M_1^2 + 2}.
\]

Substituting the above expression for $M_1^2$ into this equation and simplifying
yields a direct relation between the density and pressure ratios:
\[
\frac{\rho_2}{\rho_1} =
\frac{ \tfrac{P_2}{P_1} + \tfrac{\Gamma -1}{\Gamma +1} }
     { \tfrac{\Gamma}{\Gamma +1}\,\tfrac{P_2}{P_1} + \tfrac{1}{\Gamma +1} } .
\]
Equivalently, this can be written in a slightly cleaner form:
\begin{equation}
\boxed{\;\;
\frac{\rho_2}{\rho_1}
= \frac{ (\Gamma - 1)P_1 + (\Gamma + 1)P_2 }
       { (\Gamma + 1)P_1 + (\Gamma - 1)P_2}
\;\;}
\end{equation}
\noindent
This form of the Rankine--Hugoniot condition is particularly useful in practice:
if the pressure jump across a shock is measured (for example in X-ray
observations of galaxy clusters), the corresponding compression ratio of the
gas can be inferred directly.

\subsubsection*{The Temperature RH Conditions}

The behavior of the temperature across the RH conditions is determined from the
other ratios we have already derived. Specifically,
\[
\frac{T_2}{T_1} = \frac{P_2}{P_1} \cdot \frac{\rho_1}{\rho_2}.
\]
Substituting the expressions for the pressure and density ratios in Mach form,
\[
\frac{P_2}{P_1} = 1 + \frac{2\Gamma}{\Gamma+1}\left(M_1^2 - 1\right),
\qquad
\frac{\rho_2}{\rho_1} = \frac{(\Gamma+1)M_1^2}{(\Gamma-1)M_1^2+2},
\]
we arrive at
\[
\frac{T_2}{T_1} =
\left[\,1 + \frac{2\Gamma}{\Gamma+1}\left(M_1^2 - 1\right)\,\right]
\left[\frac{(\Gamma-1)M_1^2+2}{(\Gamma+1)M_1^2}\right].
\]
Simplifying gives the compact form
\begin{equation}
\boxed{\;\;
\frac{T_2}{T_1} =
\frac{\left[\,2\Gamma M_1^2 - (\Gamma - 1)\,\right]
      \left[(\Gamma-1)M_1^2+2\right]}
     {(\Gamma+1)^2 M_1^2}
\;\;}
\end{equation}
\noindent
As a check, in the strong-shock limit $M_1 \to \infty$,
\[
\frac{T_2}{T_1} \;\longrightarrow\;
\frac{2\Gamma(\Gamma-1)}{(\Gamma+1)^2}\,M_1^2.
\]
For a monatomic ideal gas ($\Gamma=5/3$), this reduces to
\[
\frac{T_2}{T_1} \;\longrightarrow\; \frac{5}{16}\,M_1^2.
\]

\section{Isothermal Shocks}

In the adiabatic treatment above, we neglected cooling in the energy equation. This is not always valid. When the gas is strongly coupled to a cooling process (e.g.\ radiation, conduction to a cold reservoir), thermal energy generated in the shock is \textbf{removed rapidly and the temperature remains approximately constant across the discontinuity}. Such shocks are well described by the \emph{isothermal} limit.

\begin{remark}
Adiabatic shocks apply when the cooling time is long compared to the advection time through the shock layer; isothermal shocks apply in the opposite limit:
\[
t_{\mathrm{cool}} \gg t_{\mathrm{adv}} \quad \text{(adiabatic)}, 
\qquad
t_{\mathrm{cool}} \ll t_{\mathrm{adv}} \quad \text{(isothermal)}.
\]
\end{remark}

The conservative forms of the Euler equations (1D, steady, shock at $x=0$) are
\begin{align}
&\text{Mass:} && \frac{d}{dx}(\rho u) = 0
\;\;\Longrightarrow\;\;
\rho_1 u_1 = \rho_2 u_2 \equiv m', \label{eq:iso_mass}\\[4pt]
&\text{Momentum:} && \frac{d}{dx}\!\left(\rho u^2 + p\right) = 0
\;\;\Longrightarrow\;\;
\rho_1 u_1^2 + p_1 = \rho_2 u_2^2 + p_2, \label{eq:iso_mom}\\[4pt]
&\text{Energy (with cooling):} && 
\frac{d}{dx}\!\left[u\,(E+p)\right] = -\,\mathcal{L}(x), \label{eq:energy_cooling}
\end{align}
where $E=\rho e + \tfrac{1}{2}\rho u^2$ is the total energy density and $\mathcal{L}$ is the (positive) volumetric cooling rate.

Integrating \eqref{eq:energy_cooling} across the thin shock layer gives the \emph{corrected} energy jump:
\begin{equation}
\big[u(E+p)\big]_1^2 \;=\; - \int_{-\delta x}^{+\delta x} \mathcal{L}(x)\,dx.
\label{eq:energy_jump_with_cooling}
\end{equation}
In the \textbf{isothermal limit}, cooling is efficient enough to maintain $T_2 \approx T_1 \equiv T$, so the ideal-gas equation of state reads
\begin{equation}
p = \rho c_s^2, \qquad c_s^2 \equiv \frac{k_B T}{\mu} = \text{const.}
\label{eq:isothermal_eos}
\end{equation}
In practice, one then uses \eqref{eq:iso_mass}, \eqref{eq:iso_mom}, and \eqref{eq:isothermal_eos}; the energy jump \eqref{eq:energy_jump_with_cooling} is \emph{implicitly} satisfied by the cooling that enforces $T=\text{const.}$

Insert $p=c_s^2\rho$ into the momentum jump \eqref{eq:iso_mom} and use $m'=\rho u$:
\[
\rho_1 u_1^2 + c_s^2 \rho_1 
= \rho_2 u_2^2 + c_s^2 \rho_2.
\]
With $u_i = m'/\rho_i$ this becomes
\[
\frac{m'^2}{\rho_1} + c_s^2 \rho_1
= \frac{m'^2}{\rho_2} + c_s^2 \rho_2
\;\;\Longrightarrow\;\;
m'^2\!\left(\frac{1}{\rho_1}-\frac{1}{\rho_2}\right)
= c_s^2\,(\rho_2-\rho_1).
\]
For a nontrivial jump ($\rho_2\neq\rho_1$), cancel $(\rho_2-\rho_1)$ to obtain
\begin{equation}
m'^2 = c_s^2\,\rho_1\rho_2.
\label{eq:mprime_iso}
\end{equation}
Using $m'=\rho_1 u_1$ or $m'=\rho_2 u_2$ then gives the classic isothermal relations:
\begin{align}
\frac{\rho_2}{\rho_1} 
&= \frac{u_1^2}{c_s^2} \;=\; M_1^2, \label{eq:iso_density_jump}\\[4pt]
\frac{u_2}{u_1} 
&= \frac{\rho_1}{\rho_2} \;=\; \frac{1}{M_1^2}, \label{eq:iso_velocity_jump}\\[4pt]
\frac{p_2}{p_1} 
&= \frac{\rho_2}{\rho_1} \;=\; M_1^2, \label{eq:iso_pressure_jump}
\end{align}
where $M_1 \equiv u_1/c_s$ is the \emph{isothermal Mach number} upstream. Thus, for an isothermal shock the compression (and pressure) ratio is simply $M_1^2$.

\begin{remark}
In contrast to adiabatic shocks (where the maximum compression is finite, e.g.\ $\rho_2/\rho_1 \le (\gamma+1)/(\gamma-1)$ for $\gamma>1$), an isothermal shock can in principle achieve arbitrarily large compression as $M_1$ increases. The trade-off is that the shock must radiate away the corresponding thermal energy to keep $T$ fixed.
\end{remark}

\subsection*{Cooling Length and Relevance}

Let $\mathcal{L}$ be the volumetric cooling rate and define an \emph{isobaric} or \emph{isochoric} cooling time $t_{\mathrm{cool}}$ appropriate to the downstream state (model-dependent). The \emph{cooling length} is the distance over which the post-shock flow loses the shock-generated thermal energy:
\begin{equation}
\ell_{\mathrm{cool}} \;\sim\; u_2\, t_{\mathrm{cool}}.
\end{equation}
The isothermal approximation is valid if the thermal energy is removed on a length scale short compared to the advection/dynamical scale of interest $L$:
\[
\ell_{\mathrm{cool}} \ll L
\quad\Longleftrightarrow\quad
t_{\mathrm{cool}} \ll \frac{L}{u_2}.
\]
Physically: microscopic collisions in the shock layer still convert bulk kinetic energy into random motion, but efficient cooling immediately removes that energy, preventing a temperature rise and enforcing the equation of state $p=c_s^2\rho$.


