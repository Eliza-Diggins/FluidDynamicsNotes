In this section, we will derive the critical equations of fluid dynamics: the so-called \textbf{Euler Equations}. In effect, these are derived from 3 core principles:

\begin{enumerate}
    \item \textbf{Conservation of Mass}: leads to the \emph{conservation equation}.
    \item \textbf{Conservation of Energy / Momentum}: lead to the Euler equations.
\end{enumerate}

\section{The Continuity Equation}

Consider a fluid with a conserved density field $\psi$ (generally the \textbf{mass density}). In some region $V$, the total amount of the conserved field is
\[
\Psi = \int_V \psi dV.
\]
Now, this may flow in / out of the volume or be created / destroyed within the volume. In some time $\delta t$, we describe the created / destroyed quantity as
\[
\Delta \Psi_{\rm internal} = \int_V \delta \psi dV
\]
Additionally, there is some flow in / out of the surface
\[
\Delta \Psi_{\rm flux} = dt\oint_{\partial V} \psi {\bf u} \cdot d{\bf S}.
\]
If \textbf{no material is created / destroyed}, then
\[
\frac{d\Psi}{dt} = -\oint_{\partial V} \psi \;{\bf u} \cdot d{\bf S} 
\]
By \textbf{divergence theorem},
\[
-\oint_{\partial V} {\bf X} \cdot d{\bf S} = -\int_V \nabla \cdot {\bf X} dV \implies \oint_{\partial V} \psi {\bf u} \cdot d{\bf S} = \int_V \nabla \cdot (\psi {\bf u}) dV.
\]
Therefore,
\[
\frac{d\Psi}{dt} = \int_V \frac{\partial\psi}{\partial t} dV = - \int_V \nabla \cdot (\psi {\bf u}) dV.
\]
In the limit as $V \to 0$, this becomes
\begin{equation}
\label{eq:conservation-equation-eulerian}
\boxed{
\frac{\partial\psi}{\partial t} + \nabla \cdot (\psi {\bf u}) = 0.\;\text{Eulerian.}
}
\end{equation}
\begin{remark}
    In most cases, we are chiefly concerned with the density $\rho$, in which case we have
    \[
    \frac{\partial\rho}{\partial t} + \nabla \cdot (\rho {\bf u}) = 0
    \]
    In the frequent case (in classical fluid dynamics) where $\rho$ is a constant of the flow, you can pull it out; however, this is not generally the case in compressible gas flows.
\end{remark}

To write the above result in \textbf{Lagrangian form}, we need only remember that (equation~\ref{eq:euler-lagrange-convert})
\[
\frac{\partial \psi_{\rm Eulerian}}{\partial t} = \frac{D\psi}{Dt} - {\bf u} \cdot \nabla \psi.
\]
\rmk{Recall that $\nabla \cdot (\rho {\bf u}) = \partial_k (\rho u^k) = \rho \partial_ku^k + u^k \partial_k \rho = \rho \nabla \cdot {\bf u} + {\bf u} \cdot \nabla \rho.$} Thus, substituting into equation~\ref{eq:conservation-equation-eulerian}, we find
\begin{equation}
   \label{eq:conservation-equation-lagrangian}
   \boxed{
    \frac{D\rho}{Dt} + \rho \nabla \cdot {\bf u} = 0.\;\text{Lagrangian.}}
\end{equation}
\begin{remark}
    The intuition here is not as easy to see on inspection. Imagine yourself floating along in the flow. The first term $D\rho/Dt$ describes how the \textbf{comoving density} behaves (the density you feel). Now, what should that intuitively be tied to? Well clearly to the density of the particles around you. They are all also moving with the flow, so as long as the flow lines remain totally parallel we don't get any change in density. If the flow lines converge, then the density will clearly go up as particles get pressed closer. The natural measure of that "convergence" is the \textbf{divergence of the velocity field}.

    Notice that both forms of the equation have their "problem terms." The Eulerian approach requires $\nabla \cdot (\psi{\bf u})$, which may be tricky to evaluate. Meanwhile, the Lagrangian term only requires $\nabla \cdot {\bf u}$ but also requires $D\rho/Dt$.
\end{remark}
\begin{remark}
    Consider the scenario where $D\rho/Dt = 0.$ We see immediately that this requires $\nabla \cdot {\bf u} = 0$, so the fluid is \textbf{solenoidal} (divergenceless). This is a characteristic aspect of \textbf{divergence free flows} or \textbf{incompressible flows}, which occur in many limits.
\end{remark}
\section{Vector Conservation Laws}

Just as we did for the scalar field $\psi$, we might also be interested in the conservation of a vector density field ${\bf \Psi}$ defined on the domain. This is a \textbf{trickier calculation}; however, it has a massive role in the nature of fluid dynamics, especially in the conservation of momentum.

\begin{remark}
    For a \textbf{vector density}, each component of the field must be conserved as in the scalar case; however, flux in / out of the surface in a particular index $i$ may couple with the $j$ component in the adjacent region. This was a consideration which was not necessary in the treatment of the scalar case.
\end{remark}

Consider a region $V$ of the domain with total quantity of $\psi_k$:
\[
\Psi_k = \int_V \psi_k \; dV.
\]
As in the scalar case, changes to $\Psi_k$ may arise from creation/annihilation within $V$ and flux across the boundary $\partial V$. The creation/annihilation is straightforward:
\[
\Delta \Psi_{k,\;{\rm internal}} = \int_V \frac{\partial \psi_k}{\partial t} \; dV.
\]
However, the movement across the membrane is \textbf{not as simple to describe}, since the vector field may not transport identically across all orientations of the membrane, nor in all components. We therefore introduce the \textbf{Flux Tensor} $\Sigma^k_j$, defined such that:
\[
\Delta \Psi_{k,\;{\rm flux}} = \oint_{\partial V} \Sigma^k_j \; dS^j.
\]
\rmk{Note that we have made a sign convention here which needs to be kept track of...}
Here, $dS^j$ is the oriented surface area element in the $j^{\rm th}$ direction, and $\Sigma^k_j$ expresses the flux of the $k^{\rm th}$ component of the conserved vector through surfaces with orientation in the $j^{\rm th}$ direction. The change in the total $\Psi$ in a time $\Delta t$ is thus
\[
\Delta \Psi = \int_V \frac{\partial \psi_k}{\partial t} \; dV + \oint \Sigma_j^k \;dS^j = \int_V \frac{\partial \psi_k}{\partial t} + \partial_j\Sigma_j^k\; dV \Delta t
\]
Now, the LHS provides source terms to the equation and we therefore have that, for some set of sources ${\bf f}$,
\[
\frac{\partial \boldsymbol{\psi}}{\partial t} + \nabla \cdot \boldsymbol{\Sigma} = {\bf f}.
\]
Since $V$ is arbitrary, the integrands must be equal everywhere, leading to the \textbf{general vector conservation law}:
\begin{equation}
\boxed{
\frac{\partial \psi_k}{\partial t} + \partial_j \Sigma^k_j = 0.
}
\end{equation}
In \textbf{vector notation}, this is
\[
\frac{\partial{\bf \Psi}}{\partial t} + \nabla \cdot {\bf \Sigma} = 0,\;\text{Eulerian.}
\]

\subsection{Momentum Conservation}

With the general form of a \textbf{vector conservation law} in hand, we now want to derive the explicit form of the flux tensor $\Sigma$ for the case of momentum conservation. The relevant vector density field is the \textbf{momentum density}:
\[
{\bf p} = \rho {\bf u},
\]
where $\rho$ is the mass density and ${\bf u}$ is the velocity field.

There are \textbf{two distinct physical mechanisms} by which momentum is transported in a fluid:

\begin{enumerate}
    \item \textbf{Advection}: Material crosses the boundary $\partial V$ carrying its momentum with it. The mass flux across a surface with oriented area element $dS^j$ is $\rho u_j \, dS^j$. Since each unit of mass carries momentum ${\bf p}$, the $k^{\rm th}$ component of momentum advected across the surface is:
    \[
    \Delta p^k_{\rm advective} = \rho u^k u_j \, dS^j.
    \]

    \item \textbf{Internal Stresses}: Even without bulk motion, microscopic interactions between particles on either side of the boundary may transfer momentum. These are summarized by the \textbf{stress tensor} $\sigma^k_j$, defined such that the $k^{\rm th}$ component of momentum transferred per unit area per unit time across a surface with normal $\hat{n}^j$ is:
    \[
    \Delta p^k_{\rm stress} = \sigma^k_j \, \hat{n}^j.
    \]
\end{enumerate}

Combining these mechanisms, the total momentum flux tensor takes the form:
\[
\Sigma^k_j = \rho u^k u_j + \sigma^k_j.
\]

The general vector conservation law for momentum density reads:
\[
\frac{\partial p_k}{\partial t} + \partial_j \Sigma^k_j = 0,
\]
or explicitly:
\[
\frac{\partial (\rho u_k)}{\partial t} + \partial_j \left( \rho u_k u^j + \sigma^k_j \right) = 0.
\]

In vector notation, this becomes:
\[
\frac{\partial {\bf p}}{\partial t} + \nabla \cdot ({\bf p} \otimes {\bf u}) + \nabla \cdot \boldsymbol{\sigma} = 0,
\]
where:
\begin{itemize}
    \item ${\bf p} \otimes {\bf u}$ denotes the outer product (dyadic product) of momentum and velocity, producing a rank-2 tensor with components $p^k u_j = \rho u^k u_j$,
    \item $\boldsymbol{\sigma}$ is the stress tensor.
\end{itemize}

This formulation clearly separates:
\begin{itemize}
    \item Momentum transport due to advection (${\bf p} \otimes {\bf u}$),
    \item Momentum exchange due to internal stresses ($\boldsymbol{\sigma}$).
\end{itemize}

In \textbf{some cases}, we may have \textbf{external sources of momentum}, leading us to
\[
\frac{\partial {\bf p}}{\partial t} + \nabla \cdot ({\bf p} \otimes {\bf u}) + \nabla \cdot \boldsymbol{\sigma} = {\bf F}_{\rm ext}.
\]
\begin{remark}
    We also might characterize the entire RHS as a stress tensor:
    \[
    \boldsymbol{\sigma} = \left({\bf p} \otimes {\bf u} + P{\bf I}\right),
    \]
    in which case, we have
    \[
    \frac{\partial {\bf p}}{\partial t} + \nabla \cdot \boldsymbol{\sigma} ={\bf F}_{\rm ext}.
    \]
\end{remark}
\textbf{much more can be done with a bit more work.}

\subsection{The Stress Tensor}

We have, so far, derived the critical equation
\[
\boxed{
\frac{\partial {\bf p}}{\partial t} + \nabla \cdot ({\bf p} \otimes {\bf u}) + \nabla \cdot \boldsymbol{\sigma} = {\bf F}_{\rm ext}.}
\]
for a very general $\boldsymbol{\sigma}$, these equations can be quite difficult to solve; however, we can make quite a bit of progress on physical arguments about the stress tensor. In this section, we'll take some time to explore these behaviors before proceeding.
\vspace{0.25cm}
\begin{theorem}[The stress tensor is symmetric]
\label{thrm:stress-is-symmetric}
Let ${\bf p}$ be the momentum of a particular fluid flow with velocity field ${\bf u}$ and density field $\rho$. Regardless of the nature of the system, the \textbf{stress tensor} $\sigma_{ij}$ is everywhere symmetric.
\end{theorem}
\rmk{In fact, by the same logic, you can show that ANY \textbf{flux tensor} in a momentum conservation law MUST have these properties for the same reason. This makes the flux tensor for momentum special in that it must ALWAYS be symmetric.}
\begin{proof}
The intuition behind this is relatively simple. Consider a small control volume of fluid with negligible size. If the off-diagonal elements of the stress tensor are not symmetric, the internal forces will generate a net torque proportional to the surface area of the control volume. As the size of the volume shrinks to zero, the torque remains finite, implying the possibility of unbounded angular acceleration — contradicting the assumption of local mechanical equilibrium.

We now present the formal derivation based on the conservation of angular momentum.

The total angular momentum of a region $V$ is given by:
\[
L^k = \int_V \epsilon^{kij} x_i \rho u_j \; dV,
\]
where:
\begin{itemize}
    \item $\epsilon^{kij}$ is the Levi-Civita symbol,
    \item $x_i$ is the position vector,
    \item $\rho u_j$ is the $j^{\rm th}$ component of the momentum density.
\end{itemize}

The rate of change of angular momentum is:
\[
\frac{d L^k}{dt} = \int_V \epsilon^{kij} x_i \frac{\partial (\rho u_j)}{\partial t} \; dV.
\]

Applying the momentum conservation equation:
\[
\frac{\partial (\rho u_j)}{\partial t} + \partial_m \Sigma^m_j = 0,
\]
we substitute:
\[
\frac{d L^k}{dt} = - \int_V \epsilon^{kij} x_i \partial_m \Sigma^m_j \; dV.
\]

Using the product rule and the divergence theorem:
\[
\frac{d L^k}{dt} = - \oint_{\partial V} \epsilon^{kij} x_i \Sigma^m_j \; dS_m + \int_V \epsilon^{kij} \Sigma^m_j \delta_{im} \; dV.
\]

The first term represents the torque exerted on the boundary of the region, which may be nonzero due to external forces. The second term simplifies using $\delta_{im}$:
\[
\frac{d L^k}{dt} = - \oint_{\partial V} \epsilon^{kij} x_i \Sigma^m_j \; dS_m + \int_V \epsilon^{kij} \Sigma^i_j \; dV.
\]

In the absence of internal microscopic torques (body couples), conservation of angular momentum requires that the internal torque contribution vanishes identically:
\[
\int_V \epsilon^{kij} \Sigma^i_j \; dV = 0.
\]

Since $V$ is arbitrary, the integrand itself must vanish pointwise:
\[
\epsilon^{kij} \Sigma^i_j = 0.
\]

The Levi-Civita symbol is antisymmetric in $i$ and $j$, implying that the antisymmetric part of $\Sigma^i_j$ must vanish. Explicitly, we can decompose:
\[
\Sigma^i_j = \Sigma^{(ij)} + \Sigma^{[ij]},
\]
where:
\[
\Sigma^{(ij)} = \frac{1}{2} \left( \Sigma^i_j + \Sigma^j_i \right), \quad \Sigma^{[ij]} = \frac{1}{2} \left( \Sigma^i_j - \Sigma^j_i \right).
\]

The expression $\epsilon^{kij} \Sigma^i_j = 0$ requires the antisymmetric part $\Sigma^{[ij]}$ to vanish identically. Therefore, the stress tensor is symmetric:
\[
\boxed{ \Sigma^i_j = \Sigma^j_i }.
\]

In particular, for the internal stress tensor $\sigma^i_j$, this symmetry property holds:
\[
\sigma^i_j = \sigma^j_i.
\]

Thus, the stress tensor is symmetric everywhere, consistent with conservation of angular momentum in the absence of internal body torques.

\end{proof}
\vspace{0.5cm}
Theorem~\ref{thrm:stress-is-symmetric} gives us a few very interesting corollaries:

\begin{corollary}
    For any stress tensor $\sigma_{ij}$, there is a coordinate frame (orthogonal transformation) in which $\sigma$ is \textbf{diagonalized}. In this frame, there are no shear forces and the expansion / contraction of fluid elements is more directly understood.
\end{corollary}

\begin{corollary}
    Any stress tensor may be \textbf{decomposed} into its \textbf{isotropic} and \textbf{deviatoric} components:
    
    \begin{itemize}
        \item The \textbf{isotropic component} is of the form $-p \delta_{ij}$, where $p$ is the \textbf{very familiar quantity of pressure}.
        \item The \textbf{deviatoric component} is effectively everything else: $\tau_{ij}$. It describes the shears and viscous effects.
    \end{itemize}
\end{corollary}

\subsubsection{Boundary Conditions}

In continuum mechanics, boundary conditions arise from the requirement that forces 
on surfaces balance. The relevant object is the Cauchy stress tensor 
$\boldsymbol{\sigma}$, which relates to the traction vector (force per unit area) 
acting on a surface with unit normal $\hat{\mathbf{n}}$ via
\begin{equation}
    \mathbf{t} = \boldsymbol{\sigma} \cdot \hat{\mathbf{n}}.
\end{equation}

At an interface between two media, \textbf{Newton’s third law requires} that the tractions 
exerted on the surface by each medium, together with any intrinsic surface forces, 
must balance. In the most general form:
\begin{equation}
    \big(\boldsymbol{\sigma}^{(1)} - \boldsymbol{\sigma}^{(2)}\big) \cdot \hat{\mathbf{n}}
    = \nabla_s \cdot \boldsymbol{\tau}_s,
\end{equation}
where
\begin{itemize}
    \item $\boldsymbol{\sigma}^{(1)}$, $\boldsymbol{\sigma}^{(2)}$ are the bulk stress tensors 
          on either side of the interface,
    \item $\hat{\mathbf{n}}$ is the unit normal to the interface (pointing into medium 1),
    \item $\boldsymbol{\tau}_s$ is the surface stress tensor (e.g.\ isotropic surface tension),
    \item $\nabla_s$ denotes the surface divergence operator.
\end{itemize}

This condition can be decomposed into two parts:
\begin{enumerate}
    \item \textbf{Normal stress balance:}
    \begin{equation}
        \hat{\mathbf{n}} \cdot 
        \big(\boldsymbol{\sigma}^{(1)} - \boldsymbol{\sigma}^{(2)}\big) \cdot \hat{\mathbf{n}}
        = \hat{\mathbf{n}} \cdot \big(\nabla_s \cdot \boldsymbol{\tau}_s\big),
    \end{equation}
    which determines the pressure jump across the surface (including, e.g., Laplace pressure if surface tension is present).

    \item \textbf{Tangential stress continuity:}
    \begin{equation}
        \hat{\mathbf{t}} \cdot 
        \big(\boldsymbol{\sigma}^{(1)} - \boldsymbol{\sigma}^{(2)}\big) \cdot \hat{\mathbf{n}} = 0,
    \end{equation}
    for all tangent directions $\hat{\mathbf{t}}$, unless additional surface shear stresses 
    are present.
\end{enumerate}

\medskip
\noindent
\textbf{Physical interpretation.}  
The boundary condition is simply the statement that the interface, having no bulk inertia, 
cannot sustain a net unbalanced traction. The normal component enforces pressure or normal 
stress balance, while the tangential component enforces shear stress continuity. Any intrinsic 
surface physics (e.g.\ isotropic tension, membrane elasticity) enters through the surface stress 
tensor $\boldsymbol{\tau}_s$.


\section{Euler's Equations}

We are now in a position to derive the \textbf{Euler equations} — the fundamental equations governing the motion of an \emph{inviscid} (non-viscous) fluid. These equations follow naturally by simplifying the general momentum conservation law under the assumption that viscous stresses vanish and only isotropic pressure remains.

\vspace{0.25cm}
\subsubsection*{Momentum Conservation in General Form}

Recall the momentum conservation equation:
\[
\frac{\partial {\bf p}}{\partial t} + \nabla \cdot ({\bf p} \otimes {\bf u}) + \nabla \cdot \boldsymbol{\sigma} = {\bf F}_{\rm ext},
\]
where:
\begin{itemize}
    \item ${\bf p} = \rho {\bf u}$ is the momentum density,
    \item $\boldsymbol{\sigma}$ is the internal stress tensor,
    \item ${\bf F}_{\rm ext}$ represents external body forces (e.g., gravity).
\end{itemize}

For an inviscid fluid, there are no shear or viscous stresses. The only internal stress is the isotropic pressure, so:
\[
\sigma_{ij} = -p \delta_{ij},
\]
where:
\begin{itemize}
    \item $p$ is the pressure,
    \item $\delta_{ij}$ is the Kronecker delta.
\end{itemize}

Substituting into the momentum conservation equation:
\[
\frac{\partial (\rho u_k)}{\partial t} + \partial_j \left( \rho u_k u^j \right) + \partial_j \left( -p \delta^k_j \right) = F^k_{\rm ext}.
\]

Simplifying:
\[
\frac{\partial (\rho u_k)}{\partial t} + \partial_j \left( \rho u_k u^j \right) = - \partial_k p + F^k_{\rm ext}.
\]

\subsection*{Expanding with the Continuity Equation}

We can rewrite the time derivative term using the product rule:
\[
\frac{\partial (\rho u_k)}{\partial t} = \rho \frac{\partial u_k}{\partial t} + u_k \frac{\partial \rho}{\partial t},
\]
and similarly expand the divergence term:
\[
\partial_j \left( \rho u_k u^j \right) = \rho u^j \partial_j u_k + u_k \partial_j ( \rho u^j ).
\]

Thus, the momentum equation becomes:
\[
\rho \frac{\partial u_k}{\partial t} + u_k \frac{\partial \rho}{\partial t} + \rho u^j \partial_j u_k + u_k \partial_j ( \rho u^j ) = - \partial_k p + F^k_{\rm ext}.
\]

We now use the continuity equation (conservation of mass):
\[
\frac{\partial \rho}{\partial t} + \nabla \cdot ( \rho {\bf u} ) = 0,
\]
or explicitly:
\[
\frac{\partial \rho}{\partial t} + \partial_j ( \rho u^j ) = 0.
\]

Substituting this into the momentum equation, notice:
\[
u_k \left( \frac{\partial \rho}{\partial t} + \partial_j ( \rho u^j ) \right) = 0.
\]

Therefore, the momentum equation simplifies to:
\[
\rho \left( \frac{\partial u_k}{\partial t} + u^j \partial_j u_k \right) = - \partial_k p + F^k_{\rm ext}.
\]

\subsection*{Lagrangian Form of the Euler Equations}

The left-hand side of the above is the \textbf{material derivative} of $u_k$:
\[
\frac{D u_k}{D t} = \frac{\partial u_k}{\partial t} + u^j \partial_j u_k.
\]

Thus, the \textbf{Lagrangian form} of the Euler equations is:
\begin{equation}
\label{eq:momentum_lagrangian}
\boxed{
\rho \frac{D {\bf u}}{D t} = - \nabla p + {\bf F}_{\rm ext}.
}
\end{equation}

Intuitively, this is the co-moving conservation of momentum equation, which makes a lot of sense on examination of its various terms and in analogy with the familiar form of Newton's 2nd law.

\subsection*{Eulerian Form of the Euler Equations}

Alternatively, from the stationary (Eulerian) perspective, we can simply transform the equation as
\[
\rho \partial_t {\bf u} + \rho {\bf u}\cdot \nabla {\bf u} = - \nabla p + {\bf F}_{\rm ext}.
\]

This is equivalent to 
\[
\frac{\partial {\bf p}}{\partial t} + \nabla \cdot ({\bf p} \otimes {\bf u}) + \nabla \cdot \boldsymbol{\sigma} = {\bf F}_{\rm ext},
\]
after reduction of the stress tensor and inclusion of the conservation of mass.

\begin{remark}
    In the Lagrangian viewpoint, you imagine yourself drifting along with a fluid parcel. You directly experience how the velocity of the parcel changes over time as a result of local pressure gradients and external forces, such as gravity. The material derivative $D{\bf u}/Dt$ captures this physical acceleration of the parcel.
\end{remark}

\subsection*{Summary of the Inviscid Euler Equations}

For an inviscid, compressible fluid with mass density $\rho$, velocity field ${\bf u}$, and pressure $p$, the governing equations are:

\begin{enumerate}
    \item \textbf{Conservation of Mass (Continuity Equation)}:
    \[
    \frac{\partial \rho}{\partial t} + \nabla \cdot ( \rho {\bf u} ) = 0.
    \]

    \item \textbf{Conservation of Momentum (Euler Equations)}:
    \[
    \rho \frac{D {\bf u}}{D t} = - \nabla p + {\bf F}_{\rm ext}.
    \]
\end{enumerate}

These two equations form the foundation of inviscid fluid dynamics. They are generally closed by an \textbf{equation of state} relating pressure $p$ to $\rho$ and potentially other thermodynamic quantities, especially in compressible flows.

\section{Viscosity}

So far in our discussion of momentum conservation, we have discussed only scenarios with \textbf{bulk transfer of momentum}:
\[
\frac{\partial {\bf p}}{\partial t} + \nabla \cdot ({\bf p} \otimes {\bf u}) + \nabla \cdot \boldsymbol{\sigma} = {\bf F}_{\rm ext},
\]
where $\boldsymbol{\sigma}$ is an \textbf{isotropic stress tensor} containing only the contribution of thermal pressure. In this section, we'll discuss the advent of \textbf{viscosity} and its role in the dynamics of fluids.
\par
\subsection{Intuition for Viscosity}

Viscosity originates from the \textbf{microscopic transport of momentum by
particles in thermal motion.} Even in a simple gas, molecules do not
remain confined to a single ``layer'' of the fluid: they constantly
move across notional boundaries separating adjacent regions. When a
molecule crosses such a boundary, it carries with it the momentum
characteristic of its layer of origin. If the bulk velocity differs
between neighboring layers, these molecular exchanges create a net
flux of momentum.

This picture explains why \textbf{velocity gradients} are the key
ingredient in viscous phenomena. When the flow is spatially uniform,
particle exchange across layers simply mixes identical momentum
distributions, producing no net effect. By contrast, if there is a
velocity gradient, the molecules arriving from the faster layer carry
more momentum than those arriving from the slower layer. The result is
an imbalance: momentum is transported down the gradient, and the fluid
experiences a \textbf{shear stress} opposing the relative motion of
the layers.

A simple kinetic estimate makes this idea quantitative. Consider a
shear flow with velocity $u(y)$ in the $x$--direction. Molecules
originating a distance of order the mean free path $\ell$ away bring
with them an excess (or deficit) of momentum proportional to
$\rho\,v_{\rm th}\,\ell\,\partial u/\partial y$, where $\rho$ is the
mass density and $v_{\rm th}$ the typical thermal speed. This leads to
an effective shear force per unit area
\[
\tau_{xy} \sim \mu \, \frac{\partial u}{\partial y},
\]
with a viscosity coefficient of order
\[
\mu \sim \tfrac{1}{3}\,\rho\,v_{\rm th}\,\ell.
\]
While this argument is heuristic, it captures the essential scaling
and physical mechanism: viscosity is the diffusive transport of
momentum due to microscopic motion. A more rigorous calculation using
kinetic theory (e.g.\ the Chapman--Enskog solution for a dilute gas of
hard spheres) yields a viscosity of the same order of magnitude,
providing confirmation of this simple picture.

\subsection{The Formal Theory of Viscosity}

We now develop the formal framework for viscosity. Our ultimate goal is
to build a constitutive law for viscous stresses that, when inserted into
the momentum conservation equation, yields the celebrated
Navier--Stokes equations.

\subsubsection*{The Viscous Stress Tensor}

The central physical idea is that \textbf{viscosity encodes how
\emph{velocity gradients} generate internal stresses in a fluid}.
Because viscosity represents a dissipative process, the stress must
depend only on the \emph{local deformation rate} of a fluid element.

The deformation of a fluid element is described by the velocity gradient
tensor,
\[
V_{\mu\nu} \equiv \frac{\partial u_\mu}{\partial x_\nu}.
\]
This tensor can be decomposed into its symmetric and antisymmetric
parts:
\[
V_{\mu\nu} = S_{\mu\nu} + A_{\mu\nu},
\]
where $S_{\mu\nu}$ is the symmetric \emph{rate-of-strain tensor} and
$A_{\mu\nu}$ is the antisymmetric \emph{vorticity tensor}. The latter
corresponds to a rigid-body rotation, which does not distort the shape
of the fluid element and therefore cannot produce internal stresses.
Only the symmetric part $S_{\mu\nu}$ contributes to viscous effects.

Motivated by these considerations, we postulate the most general linear
constitutive relation between stress and strain rate:
\[
\tau_{ij} = \mu_{ij\mu\nu}\,S_{\mu\nu},
\]
where $\tau_{ij}$ is the viscous stress tensor and
$\mu_{ij\mu\nu}$ is the rank-4 \emph{viscosity tensor} that encodes the
proportionality between strain rates and stresses. In full generality, $\mu_{ijkl}$ could contain $3^4=81$ components in three dimensions. Physical symmetries reduce this number dramatically. There are a number of such symmetries; however, the following are the most relevant in the theory of viscous fluids:
\vspace{0.5cm}
\paragraph{1. Stress symmetry.}
Because $\tau_{ij} = \tau_{ji}$, only the part of $\mu_{ijkl}$ symmetric under $i \leftrightarrow j$ contributes. Similarly, since $S_{kl}$ is symmetric, only the part symmetric under $k \leftrightarrow \ell$ is relevant.

\paragraph{2. Isotropy.}
An isotropic fluid must respond identically in all directions. Formally, this means $\mu_{ijkl}$ must remain invariant under any rotation:
\[
\mu_{ijkl} = R_{ip}R_{jq}R_{kr}R_{\ell s}\,\mu_{pqrs}, \qquad \forall R \in SO(3).
\]
The only building blocks of such isotropic tensors are Kronecker deltas. To construct a rank-4 isotropic tensor, we pair indices in all inequivalent ways.

\paragraph{3. Independent delta pairings.}
There are exactly three independent pairings:
\[
\delta_{ij}\delta_{k\ell}, \qquad
\delta_{ik}\delta_{j\ell}, \qquad
\delta_{i\ell}\delta_{jk}.
\]
Thus the most general isotropic form is
\[
\mu_{ijkl} = \alpha \,\delta_{ij}\delta_{k\ell}
+ \beta \,\delta_{ik}\delta_{j\ell}
+ \gamma \,\delta_{i\ell}\delta_{jk},
\]
with scalar coefficients $\alpha,\beta,\gamma$. (\rmk{Think about acting on the $\delta$-based $\mu$ with the various rotation matrices. It is easy to convince yourself that this is one way to preserve the tensor. Then each of the possible ways to create a pairing can be present.})

\paragraph{4. Reduction.}
Contracting this form with $S_{kl}$ yields
\[
\tau_{ij} = \alpha \,\delta_{ij} S_{kk} + (\beta + \gamma) S_{ij}.
\]
This shows that the viscous stress is determined by only two independent constants:
\[
\tau_{ij} = 2\mu S_{ij} + \lambda \,\delta_{ij} S_{kk},
\]
where $\mu = \beta + \gamma$ and $\lambda = \alpha$.

\subsection{Shear and Bulk Viscosity}

We have shown above that the viscous stress tensor takes the form
\[
\tau_{ij} = 2\mu S_{ij} + \lambda \delta_{ij} S_{kk},
\]
however, we may get an even more intuitive expression where things begin to reflect some degree of \textbf{physical structure}. We first recognize that $S_{ij}$ has \textbf{diagonal elements } $S_{ii}$ which correspond to \textbf{contraction and expansion}. Likewise, it has \textbf{off-diagonal elements} $S_{ij}$ corresponding to \textbf{shearing}. What we can do is break $\boldsymbol{S}$ into these two parts:
\vspace{0.25cm}
\begin{enumerate}
    \item The \textbf{isotropic component}: contains the uniform expansion / contraction of the material,
    \item The \textbf{traceless (deviatoric) component}: which encodes the shear.
\end{enumerate}
\vspace{0.25cm}
Formally, we write
\[
S_{ij} = \underbrace{\frac{1}{3} S_{kk} \delta_{ij}}_{\text{Isotropic Component}} + \underbrace{\left(S_{ij} - \frac{1}{3} S_{kk}\delta_{ij}\right)}_{\text{traceless component}}.
\]
\par
If we now propogate this expansion into the \textbf{viscous stress}, we find
\begin{equation}
    \tau_{ij} = \underbrace{2\mu S'_{ij}}_{\text{Deviatoric Component}} + \underbrace{\left(\frac{2}{3}\mu + \lambda\right)\delta_{ij} S_{kk}}_{\text{Trace Component}}.
\end{equation}
If we now combine our constants into a nice form, we can write
\begin{equation}
    \label{eq:viscous_stress_tensor}
    \boxed{
    \tau_{ij} = \underbrace{2\mu S'_{ij}}_{\text{Shear Viscous Stresses}} + \underbrace{\zeta \delta_{ij} S_{kk}}_{\text{Bulk Viscosity}},
    }
\end{equation}
where we have introduced (or reinterpretted in the case of $\mu$) the constants to be the coefficients of \textbf{bulk} and \textbf{shear} viscosity respectively. We now state this result with definitions:
\vspace{0.5cm}
\begin{definition}[Shear Viscosity]\label{def:shear-viscosity}
The \textbf{shear viscosity} $\mu$ quantifies the internal resistance of a fluid to \emph{shearing deformations}, i.e.~motions where parallel fluid layers slide past one another at different velocities. It governs the stresses proportional to the traceless, symmetric part of the strain-rate tensor. Everyday examples include the drag experienced when stirring honey or the laminar shear between plates in Couette flow.
\end{definition}

\begin{definition}[Bulk Viscosity]\label{def:bulk-viscosity}
The \textbf{bulk viscosity} $\lambda$ quantifies the fluid's resistance to \emph{volumetric deformations}, i.e.~uniform compression or expansion. It appears as an isotropic stress proportional to $\nabla \cdot \mathbf{u}$. For incompressible flows ($\nabla \cdot \mathbf{u} = 0$), bulk viscosity plays no role. In compressible flows, however—particularly when internal degrees of freedom such as molecular vibrations or rotations are excited—bulk viscosity can strongly affect energy dissipation.
\end{definition}


\subsection{The Navier-Stokes Equations}
We are now prepared to derive the most famous equation in fluid dynamics: the \textbf{Navier-Stokes equation}. This equation is the equivalent to the Euler equation when we include viscous stresses. Let's consider a system with \textbf{bulk viscosity} $\zeta$ and \textbf{shear viscosity} $\mu$. The stress tensor $\boldsymbol{\sigma}$ will take the form
\[
\boldsymbol{\sigma} = -P{\bf I} + \boldsymbol{\tau}.
\]
Now, as we derived above \eqref{eq:viscous_stress_tensor},
\[
    \tau_{ij} = \underbrace{2\mu S'_{ij}}_{\text{Shear Viscous Stresses}} + \underbrace{\zeta \delta_{ij} S_{kk}}_{\text{Bulk Viscosity}},
\]
This form is \textbf{not particularly conducive to manipulation}, as such, we'll use the form without splitting into the deviatoric and bulk terms, but this time, we'll replace $\lambda$ with $\zeta$:
\[
\tau_{ij} = 2\mu S_{ij} + \left(\zeta - \frac{2}{3}\mu\right) \delta_{ij} S_{kk} = \mu(u_{i;j} + u_{j;i}) + \left(\zeta-\frac{2}{3}\mu\right) \delta_{ij}u_{k;k},
\]
where the $;$ is the \textbf{covariant derivative}. In vector notation,
\[
\boldsymbol{\tau} = \mu\left[\nabla {\bf u} + \nabla{\bf u}^T\right] + {\bf I}\left(\zeta-\frac{2}{3}\mu\right)\nabla \cdot {\bf u}.
\]
As such, we have
\[
\sigma_{ij;j} = - P_{;i} + \tau_{ij;j} = -P_{;i} + \mu\left(u_{i;jj} + u_{j;ij}\right) + \left(\zeta - \frac{2}{3}\mu\right)\delta_{ij} u_{k;ki}.
\]
Because the order of operations doesn't matter, this becomes
\[
\sigma_{ij;j} = - P_{;i} + \tau_{ij;j} = -P_{;i} + \mu u_{i;jj} + \left(\zeta + \frac{1}{3}\mu\right)\delta_{ij} u_{k;ki}.
\]
\rmk{Note the manipulation of the last term...}
In vector notation, this is
\[
\nabla \cdot \boldsymbol{\sigma} = -\nabla P + \mu \nabla^2 {\bf u} + \left(1+\frac{1}{3}\right)\nabla(\nabla \cdot {\bf u}).
\]
\noindent
Starting from the conservative momentum equation in terms of momentum density 
${\bf p} = \rho {\bf u}$,
\begin{equation}
\frac{\partial {\bf p}}{\partial t} + \nabla \cdot ({\bf p} \otimes {\bf u}) + \nabla \cdot \boldsymbol{\sigma} = {\bf F}_{\rm ext},
\end{equation}
Hence, we arrive at \textbf{the conservative form of the Navier--Stokes momentum equation} becomes
\begin{equation}
\label{eq:navier_stokes_momentum_conservative}
\boxed{
\frac{\partial (\rho \mathbf{u})}{\partial t}
+ \nabla \cdot (\rho \mathbf{u}\otimes \mathbf{u})
= -\nabla P + \mu \nabla^2 \mathbf{u}
+ \left(\zeta + \tfrac{1}{3}\mu\right)\nabla(\nabla \cdot \mathbf{u})
+ {\bf F}_{\rm ext}.}
\end{equation}
In its \textbf{convective (non--conservative) form}, we divide through by $\rho$ and use the material derivative
\[
\frac{D}{Dt} = \frac{\partial}{\partial t} + \mathbf{u}\cdot\nabla,
\]
so that the momentum equation becomes
\begin{equation}
\label{eq:navier_stokes_momentum_convective}
\boxed{
\rho \frac{D \mathbf{u}}{Dt}
= -\nabla P + \mu \nabla^2 \mathbf{u}
+ \left(\zeta + \tfrac{1}{3}\mu\right)\nabla(\nabla \cdot \mathbf{u})
+ {\bf F}_{\rm ext}.}
\end{equation}

\noindent
In the limit of \textbf{incompressible flow}, the density is constant and the velocity field is solenoidal, $\nabla\cdot \mathbf{u} = 0$. In this case, the bulk viscosity term vanishes and the equation reduces to
\begin{equation}
\label{eq:navier_stokes_incompressible}
\boxed{
\rho \left(\frac{\partial \mathbf{u}}{\partial t} + \mathbf{u}\cdot\nabla \mathbf{u}\right)
= -\nabla P + \mu \nabla^2 \mathbf{u} + {\bf F}_{\rm ext},}
\end{equation}
subject to the incompressibility condition
\begin{equation}
\nabla \cdot \mathbf{u} = 0.
\end{equation}
\medskip
\noindent
Equations~\eqref{eq:navier_stokes_momentum_conservative}--\eqref{eq:navier_stokes_incompressible}, together with the continuity equation and an appropriate energy equation or equation of state, constitute the \textbf{Navier--Stokes system} of fluid dynamics.

\section{Surfaces and Boundaries}

\textcolor{red}{This needs to be written}

Rigid boundaries (no-slip, slip).

Free surfaces (stress-free + material).

Fluid–fluid interfaces (stress balance between two fluids).

Porous/permeable boundaries (flux across).

Phase-change interfaces (flux + stress balance).

Artificial/far-field boundaries (boundary conditions at infinity or domain edges).

